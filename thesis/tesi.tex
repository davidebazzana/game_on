% !TeX encoding = UTF-8
% !Tex program = Texmaker
% !Tex spellcheck = it_IT

\documentclass[binding=0.6cm,Lau]{sapthesis}

\usepackage{microtype}
\usepackage[italian]{babel}
\usepackage[utf8]{inputenc}
\usepackage{hyperref}
\hypersetup{pdftitle={La mia tesi},pdfauthor={Francesco Biccari}}

\title{La mia tesi}
\author{Giacomo Colizzi Coin}
\IDnumber{1794538}
\course{Ingegneria informatica}
\courseorganizer{Facoltà di ingegneria dell'informazione, informatica e statistica}
\AcademicYear{2019/2020}
\copyyear{2020}
\advisor{Prof. Querzoni}
\authoremail{colizzicoin.1794538@studenti.uniroma1.it}

\begin{document}
\frontmatter
\maketitle

\begin{abstract}
Questa è la mia tesi
\end{abstract}

\tableofcontents

\mainmatter
\chapter{Introduzione}

Nella società attuale, programmare sta diventando sempre più comune e
alla portata di tutti.
% Inserire statistiche qui
Uno degli ambiti di interesse più comuni nella fascia di età citata è
quello dei videogiochi. Esistono molte piattaforme in cui un utente
può scegliere un gioco e giocarci liberamente, ed esistono anche
piattaforme che permettono di caricare i propri giochi. Queste
piattaforme richiedono però di avere dei requisiti particolari.
Inoltre, la velocità sempre crescente delle connessioni internet ha
permesso ai servizi di streaming di avere sempre più successo.

In questo contesto nasce l'idea di questa tesi.
Il progetto ha come obiettivo la creazione di una piattaforma per la
condivisione di videogiochi. Gli utenti di questa piattaforma possono
condividere con gli altri utenti i giochi da loro sviluppati. Tutti i
giochi caricati sulla piattaforma sono giocabili da ogni utente.
Infine, come utente, è possibile condividere le proprie esperienze di
gioco votando, commentando, invitando amici e altro.

\backmatter
\cleardoublepage
\phantomsection
\addcontentsline{toc}{chapter}{\bibname}
	
\end{document}
