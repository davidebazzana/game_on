\chapter{Requisiti, tecnologie e metodologie}

\section{Funzionalità offerte}

Come detto in precedenza, il sistema informativo progettato intende offrire agli utenti la possibilità di condividere, modificare e giocare ai videogiochi da loro sviluppati.\\
Chiunque voglia usufruire di questo servizio può creare un proprio account all'interno del sistema. I giochi caricati saranno visibili e giocabili da tutti gli altri utenti.\\
Per entrare nello specifico, passiamo ora alla descrizione delle funzionalità principali. Per comodità di esposizione, nella descrizione di queste, divideremo l'analisi in sezioni e faremo riferimento diretto alle \textit{user stories}.

\subsection{Sezione utenti}

Un visitatore del sito web che intende accedere al servizio deve ottenere l'autorizzazione del sistema creando un account personale. Alla registrazione vengono richiesti uno \textit{username}, un'\textit{email} valida e una \textit{password} per gli accessi futuri. In alternativa può registrarsi attraverso l'utilizzo della tecnologia \textit{OAuth}. Da questo momento in poi sarà definito \textit{giocatore}. Dopo aver confermato la sua registrazione, potrà accedere nuovamente al servizio fornendo le proprie credenziali. Avrà anche la possibilità di modificare le informazioni relative al proprio account.

Le user stories relative agli utenti sono le seguenti:
\textit{
  \begin{enumerate}
    \item Come visitatore, voglio creare un nuovo account, in modo che il sistema possa ricordarsi di me.
    \item Come giocatore, voglio effettuare il login, in modo che possa interagire col sistema.
    \item Come giocatore, voglio poter effettuare l'accesso con un account Google o Github.
    \item Come giocatore, voglio poter terminare la sessione, così che il mio account sia protetto da usi non autorizzati.
    \item Come giocatore, voglio poter modificare i dati del mio account, così che lo possa mantenere aggiornato.
  \end{enumerate}
}

\begin{figure}[h!]
  \centering
  \includegraphics[width=0.45\textwidth]{mockup_registration}\includegraphics[width=0.45\textwidth]{mockup_login}
  \caption{Mockup relativi alle user stories 1, 2 e 3}
\end{figure}

\subsection{Sezione giochi}

Ogni utente (anche un visitatore) può vedere l'elenco dei giochi presenti sulla piattaforma e effettuare ricerche all'interno di esso, mentre solo i giocatori possono giocare, caricare giochi e modificare o eliminare i propri. Per caricare un suo gioco sviluppato con \textit{Unity} un giocatore deve compilare il form presente in una pagina dedicata. Ogni giocatore può esprimere una preferenza su un gioco o fornire un voto positivo o negativo. Può inoltre vedere la lista dei suoi giochi preferiti.

Le user stories relative a queste funzionalità sono le seguenti:
\textit{
  \begin{enumerate}
    \item Come utente generico, voglio visualizzare l'elenco di tutti i giochi aggiunti nella piattaforma dagli altri giocatori.
    \item Come utente generico,
          voglio cercare un gioco specifico cercandolo per nome o per categoria, in modo che possa cercare solo ciò che mi interessa.
    \item Come utente generico,
          voglio ordinare i risultati di una ricerca, in modo che possa avere i risultati disposti secondo vari criteri.
    \item Come giocatore, voglio inserire un gioco nella piattaforma, in modo che sia visibile agli altri utenti.
    \item Come giocatore, voglio rimuovere un mio gioco dalla piattaforma, in modo che non sia più disponibile su di essa.
    \item Come utente generico, voglio vedere le informazioni di un gioco.
    \item Come giocatore, voglio giocare a un gioco.
    \item Come giocatore proprietario di un gioco, voglio modificare il gioco, così che lo possa mantenere aggiornato.
    \item Come giocatore proprietario di un gioco, voglio rilasciare una patch del gioco, così da mostrare solo la versione più aggiornata.
    \item Come giocatore, voglio assegnare o rimuovere un like o un dislike ad un gioco, in modo da fornire un feedback pubblico allo sviluppatore.
    \item Come giocatore, voglio aggiungere/rimuovere un gioco ai miei preferiti, in modo che lo possa trovare nella mia lista preferiti.
    \item Come giocatore, voglio visualizzare una pagina con tutti i miei giochi preferiti, in modo che siano tutti raggruppati in un unico posto.
  \end{enumerate}
}

\begin{figure}[h!]
  \centering
  \includegraphics[width=0.45\textwidth]{mockup_games}
  \includegraphics[width=0.45\textwidth]{mockup_add_game}
  \caption{Mockup relativi alle user stories 1 e 4}
\end{figure}

\subsection{Sezione interazione tra utenti}

Sula piattaforma è possibile interagire con altri utenti in diversi modi. Ogni giocatore può aggiungere un commento ad un gioco che sarà successivamente visibile a tutti gli altri giocatori. Può seguire altri giocatori che compariranno nella lista dei \textit{seguiti}. Se due giocatori si seguono a vicenda sono considerati amici e possono scambiarsi inviti per giocare ad un gioco specifico. Inoltre un giocatore può contattare il creatore di un gioco per segnalare eventuali bug.

Le user stories relative all'interazione tra utenti sono le seguenti:
\textit{
  \begin{enumerate}
    \item Come giocatore, voglio vedere una lista degli altri giocatori, in modo da trovare qualcuno da seguire.
    \item Come giocatore, voglio seguire un altro giocatore, in modo che sappia il mio interesse per i suoi contenuti.
    \item Come giocatore, voglio vedere una lista di tutti i giocatori che seguo, in modo da sapere chi è offline e chi è online.
    \item Come giocatore, voglio smettere di seguire un altro giocatore, in modo che non compaia più nella lista dei giocatori che seguo.
    \item Come giocatore, voglio vedere una lista di tutti i giocatori che mi seguono, in modo da sapere chi ha interesse per i miei contenuti.
    \item Come giocatore, voglio invitare un amico a giocare ad un gioco specifico, in modo che il mio amico giochi allo stesso gioco a cui sto giocando io.
    \item Come giocatore, voglio inviare una segnalazione di bug al proprietario di un gioco sulla piattaforma, in modo che riceva un'email con la segnalazione.
    \item Come giocatore, voglio inserire un commento ad un gioco, in modo che il mio commento sia visibile a tutti.
    \item Come giocatore, voglio vedere una lista di tutti i commenti ad un gioco, in modo da sapere cosa ne pensano altri giocatori.
    \item Come giocatore, voglio eliminare un mio commento ad un gioco, in modo che il commento non sia più presente sulla piattaforma.
  \end{enumerate}
}

\begin{figure}[h!]
  \centering
  \includegraphics[width=0.45\textwidth]{mockup_lista_utenti}
  \includegraphics[width=0.45\textwidth]{mockup_lista_amici}
  \caption{Mockup relativi alle user stories da 1 a 5}
\end{figure}

\begin{figure}[h!]
  \centering
  \includegraphics[width=0.45\textwidth]{mockup_add_review}
  \includegraphics[width=0.45\textwidth]{mockup_reviews}
  \caption{Mockup relativi alle user stories 8, 9 e 10}
\end{figure}

\subsection{Sezione amministratori e moderatori}

Alcuni giocatori avranno dei ruoli particolari. I moderatori possono eliminare qualsiasi gioco e commento pubblicati da qualsiasi utente. Gli amministratori hanno tutti i privilegi dei moderatori e in più possono eliminare l'account di qualsiasi utente dalla piattaforma. I giocatori possono contattare gli amministratori per fornire un feedback riguardo alla piattaforma.

Le user stories relative a moderatori e amministratori sono le seguenti:
\textit{
  \begin{enumerate}
    \item Come moderatore, voglio eliminare un qualsiasi gioco, in modo che la piattaforma non abbia giochi che non rispettano le regole della community.
    \item Come moderatore, voglio eliminare un qualsiasi commento ad un gioco, in modo che la piattaforma non contenga commenti che violano le norme della community.
    \item Come amministratore, voglio eliminare l'account di un giocatore, in modo che non sia più presente sulla piattaforma.
    \item Come giocatore, voglio inviare un messaggio agli amministratori, in modo da segnalare bug o il comportamento di altri utenti registrati.
  \end{enumerate}
}

\section{Tecnologie utilizzate}

\subsection{Il framework \textit{Ruby on Rails}}
Per la realizzazione del progetto è stato usato il framework Ruby on Rails\cite{RoR}\cite{docror}, uno dei framework più potenti e popolari per lo sviluppo di
applicazioni web dinamiche.
Questo framework è molto utile perché permette di implementare in maniera facile il paradigma \textit{Model-View-Controller} legando per convenzioni le varie componenti lasciando lo sviluppatore libero da meccanicismi di programmazione. Ruby on Rails è anche strettamente legato ai principi \textit{REpresentational State Tranfer} (REST) per i sistemi distribuiti, una caratteristica quasi fondamentale per una moderna applicazione web.

\subsection{Software e servizi aggiuntivi}
Il linguaggio Ruby può essere integrato con numerose librerie che prendono il nome di \textit{gemme}. Nella realizzazione del progetto ne sono state utilizzate diverse, alcune trattate approfonditamente nei corsi, altre meno.

Le seguenti sono le più significative:
\begin{itemize}
  \item \textbf{Acts As Votable} è una gemma che ha come semplice obiettivo quello di permettere a qualsiasi \textit{modello} di votare o esprimere una preferenza verso un qualsiasi altro modello dell'applicazione e prevede una sintassi di utilizzo molto semplice. Questa gemma è stata usata per realizzare la funzionalità di like, dislike dei giochi da parte dei giocatori.
  \item \textbf{Clamby} è una gemma che permette di scansionare un file per verificare l'eventuale presenza di virus al suo interno sfruttando l'antivirus \textit{Clamav}. Quando i file di build di un gioco vengono caricati sulla piattaforma vengono scansionati per evitare che gli utenti possano eseguire codice malevolo.
  \item \textbf{OmniAuth Google OAuth2} e \textbf{OmniAuth GitHub} sono due gemme che consentono di effettuare il login attraverso meccanismi di OAuth con il proprio account Google o GitHub.
  \item \textbf{CanCanCan} ha permesso la definizione dei ruoli e la gestione delle autorizzazioni in base a essi.
  \item \textbf{Devise} è stata usata per gestire gli utenti e il sistema di autenticazione principale.\cite{git}
\end{itemize}
Un servizio esterno utilizzato dall'applicazione è \textit{TypingDNA}. TypingDNA mette a disposizioni una serie di \textit{API REST} che forniscono un ulteriore livello di sicurezza sfruttando i parametri di \textit{Behavioral Biometrics}, cioè quelle peculiarità comportamentali che ogni persona possiede e che la identificano. Il funzionamento e l'utilizzo di questo servizio verranno spiegati più approfonditamente nel capitolo 7.

\section{Metodologia utilizzata di conduzione del progetto ed organizzazione del lavoro}

Il gruppo di lavoro è formato da tre componenti: Davide Bazzana,
Giacomo Colizzi Coin e Renato Giamba.\\

Lo sviluppo del progetto è stato organizzato seguendo la metodologia
ASD (\textit{agile software development}).\cite{saas}

Il calendario è stato suddiviso in iterazioni, in modo da definire
quante e quali user stories dovevano essere sviluppate in un certo
periodo di tempo. Subito dopo aver definito le user stories da
svluppare, esse sono state suddivise in gruppi, ciascuno dei quali affidato ad
una iterazione. Nella prima parte dello sviluppo, la durata di
ciascuna iterazione è stata di 3 settimane, successivamente però tale
durata è stata ridotta a 2 settimane, per accelerare lo sviluppo e
concludere il progetto rispettando la data di consegna che il gruppo
si era inizialmente proposto.  Ad ogni scadenza di una iterazione, ed
inizio di quella successiva, il gruppo si riuniva per discutere dei
traguardi raggiunti e di quelli da raggiungere in futuro. Lo scopo
principale di queste riunioni era infatti quello di assegnare a
ciascun componente del gruppo un insieme di user stories appartenenti
alla successiva iterazione.  Questa scelta veniva fatta in base alla
propensione dei singoli componenti e alle competenze di ciascuno di
essi, sia acquisite in altre occasioni, sia acquisite durante lo
sviluppo del progetto stesso. Questo modus operandi ha fatto sì che i
vari componenti del gruppo, pur continuando a mantenere una visione
chiara del funzionamento complessivo dell'applicazione, si siano
specializzati in un suo ambito specifico.

\subsection{Servizi di terze parti utilizzati}

Per gestire ed analizzare l'andamento dello sviluppo del progetto è
stato utilizzato il tool \textit{PivotalTracker}.

Questo tool ha permesso al gruppo di tenere traccia di tutte le user
stories e delle loro relative informazioni: il contenuto della user
story, il suo stato di sviluppo (unstarted, started, finished, delivered,
rejected, accepted), l'iterazione alla quale faceva riferimento, il
componente responsabile del suo sviluppo, il livello di difficoltà, ed
altro. Per tutta la durata del progetto, ciascun componente aveva la
possibilità di consultare e modificare tali informazioni.

Infine, PivotalTracker ha reso possibile valutare in tempo
reale le informazioni relative all'andamento dello sviluppo, come la
sua velocità e volatilità.\\

Per il controllo di versione è stato scelto il software \textit{Git},
mentre, come servizio di hosting per la repository del progetto, è
stato scelto GitHub.

La metodologia impiegata dal gruppo per l'utilizzo dei branch della
repository è stata quella di creare un branch per ciascuna user story
in sviluppo e di mantenere un branch chiamato ``\textit{master}'' come
branch di mantenimento di una versione stabile.

La procedura di controllo del nuovo software prodotto invece, è stata quella
di \textit{peer review}. Infatti, ogni volta che un componente
del gruppo completava l'implementazione di una user story, sottoponeva
il suo lavoro al giudizio degli altri. Questa procedura veniva svolta
facendo uso del meccanismo di pull request offerto da
GitHub. Se il resto dei componenti del gruppo riteneva il
lavoro svolto adeguato, la \textit{pull request} veniva accettata e si
procedeva col \textit{merging} del branch in questione con il branch
``\textit{master}'', altrimenti, il componente, o i componenti, che
non ritenevano il lavoro svolto sufficiente, commentavano la
pull request evidenziando eventuali errori da correggere o
miglioramenti da apportare.\\

Infine è stata sfruttata la possibilità di integrare un progetto su
PivotalTracker con una repository su GitHub.

Con questa funzionalità abilitata, tutti gli avanzamenti di un branch
relativo ad una user story venivano comunicati da GitHub a
PivotalTracker, provocando così l'aggiornamento automatico
dello stato della user story su PivotalTracker.

\subsection{Statistiche di sviluppo}

Il progetto è iniziato il 21 marzo 2020 e finito il 16 settembre 2020, con una durata totale di 25 settimane e 5 giorni. L'aggiunta della funzionalità relativa a TypingDNA è stata realizzata nelle settimane successive.\\

La tabella seguente mostra alcune statistiche sul contributo di
ciascun componente del gruppo. Pur riportando il numero esatto di
commit, file e righe prodotte da ciascuno, i seguenti dati sono
indicativi. Infatti, ad esempio, una percentuale non indifferente del
numero di file e linee indicate, non costituisce un contributo diretto
del componente, bensì deriva dalle gemme e dai file di terze parti
utilizzati, come quelli prodotti da \textit{Unity} per il
gioco di test ``\textit{Kloby}''.

\begin{table}[h!]
  \centering
  \begin{tabular}{ m{0.20\textwidth} | m{0.16\textwidth} | m{0.16\textwidth} | m{0.16\textwidth} | m{0.16\textwidth} }
    \textbf{Sviluppatore} & \textbf{Commit} & \textbf{File cambiati} & \textbf{Linee inserite} & \textbf{Linee eliminate} \\
    \hline
    Davide Bazzana        & 95              & 269                    & 34507                   & 29694                    \\
    \hline
    Giacomo Colizzi Coin  & 79              & 654                    & 16085                   & 11596                    \\
    \hline
    Renato Giamba         & 28              & 182                    & 1241                    & 185                      \\
    \hline
  \end{tabular}
\end{table}
