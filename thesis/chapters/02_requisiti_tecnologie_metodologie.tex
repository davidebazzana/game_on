\chapter{Requisiti, tecnologie e metodologie}

\section{Metodologia utilizzata di conduzione del progetto ed organizzazione del lavoro}

Il gruppo di lavoro è formato da tre componenti: Davide Bazzana,
Giacomo Colizzi Coin e Renato Giamba.
\newline
\newline
Lo sviluppo del progetto è stato organizzato seguendo la metodologia
ASD (\textit{agile software development}).

Il calendario è stato suddiviso in iterazioni, in modo da definire
quante e quali user stories dovevano essere sviluppate in un certo
periodo di tempo. Subito dopo aver definito le user stories da
svluppare, esse sono state suddivise in gruppi, ciascuno affidato ad
una iterazione. Nella prima parte dello sviluppo, la durata di
ciascuna iterazione è stata di 3 settimane, successivamente però tale
durata è stata ridotta a 2 settimane, per accelerare lo sviluppo e
concludere il progetto rispettando la data di consegna che il gruppo
si era inizialmente proposto.  Ad ogni scadenza di una iterazione, ed
inizio di quella successiva, il gruppo si riuniva per discutere dei
traguardi raggiunti e di quelli da raggiungere in futuro. Lo scopo
principale di queste riunioni era infatti quello di assegnare a
ciascun componente del gruppo un insieme di user stories appartenti
alla successiva iterazione.  Questa scelta veniva fatta in base alla
propensione dei singoli componenti e alle competenze di ciascuno di
essi, sia acquisite in altre occasioni, sia acquisite durante lo
sviluppo del progetto stesso. Questo modus operandi ha fatto sì che i
vari componenti del gruppo, pur continuando a mantenere una visione
chiara del funzionamento complessivo dell'applicazione, si siano
specializzati in un suo ambito specifico.

\subsection{Servizi di terze parti utilizzati}

Per gestire ed analizzare l'andamento dello sviluppo del progetto è
stato utilizzato il tool \textit{PivotalTracker}.

Questo tool ha permesso al gruppo di tenere traccia di tutte le user
stories e delle loro relative informazioni: il contenuto della user
story, il suo stato di sviluppo (unstarted, started, finished, delivered,
rejected, accepted), l'iterazione alla quale faceva riferimento, il
componente responsabile del suo sviluppo, il livello di difficoltà, ed
altro. Per tutta la durata del progetto, ciascun componente aveva la
possibilità di consultare e modificare tali informazioni.

Infine, \textit{PivotalTracker}, ha reso possibile valutare in tempo
reale le informazioni relative all'andamento dello sviluppo, come la
sua velocità e volatilità.
\newline
\newline
Per il controllo di versione è stato scelto il software \textit{Git},
mentre, come servizio di hosting per la repository del progetto, è
stato scelto \textit{GitHub}.

La metodologia impiegata dal gruppo per l'utilizzo dei branch della
repository è stata quella di creare un branch per ciascuna user story
in sviluppo e di mantenere un branch chiamato ``\textit{master}'' come
branch di mantenimento di una versione stabile.

La procedura di controllo del nuovo software prodotto invece, è stata
quella di \textit{peer review}. Infatti, ogni volta che un componente
del gruppo completava l'implementazione di una user story, sottoponeva
il suo lavoro al giudizio degli altri. Questa procedura veniva svolta
facendo uso del meccanismo di pull request offerto da
\textit{GitHub}. Se il resto dei componenti del gruppo riteneva il
lavoro svolto adeguato, la \textit{pull request} veniva accettata e si
procedeva col \textit{merging} del branch in questione con il branch
``\textit{master}'', altrimenti, il componente, o i componenti, che
non ritenevano il lavoro svolto sufficiente, commentavano la
\textit{pull request} evidenziando eventuali errori da correggere o
miglioramenti da apportare.
\newline
\newline
Infine è stata sfruttata la possibilità di integrare un progetto su
\textit{PivotalTracker} con una repository su \textit{GitHub}.

Con questa funzionalità abilitata, tutti gli avanzamenti di un branch
relativo ad una user story venivano comunicati da \textit{GitHub} a
\textit{PivotalTracker}, provocando così l'aggiornamento automatico
dello stato della user story su \textit{PivotalTracker}.

\subsection{Statistiche di sviluppo}

Il progetto è iniziato il 21 marzo 2020 e finito il 16 settembre
2020, con una durata totale di 25 settimane e 5 giorni.
\newline
\newline
La tabella seguente mostra alcune statistiche sul contributo di
ciascun componente del gruppo. Pur riportando il numero esatto di
commit, file e righe prodotte da ciascuno, i seguenti dati sono
indicativi. Infatti, ad esempio, una percentuale non indifferente del
numero di file e linee indicate, non costituisce un contributo diretto
del componente, bensì deriva dalle gemme e dai file di terze parti
utilizzati, come quelli prodotti da \textit{Unity 3D} per il
gioco di test ``\textit{Kloby}''.

\vspace{1cm}
\begin{tabular}{l|r|r|r}
  \textbf{Developer} & \textbf{Commits} & \textbf{Files changed} & \textbf{Total lines} \\
  \hline
  Davide Bazzana & 111 & 366 & 7428 \\
  \hline
  Giacomo Colizzi Coin & 88 & 506 & 2680 \\
  \hline
  Renato Giamba & 42 & 183 & 1110 \\
  \hline
\end{tabular}
