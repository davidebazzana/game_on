\chapter{Introduzione}

Nella società attuale, programmare sta diventando sempre più comune e
alla portata di tutti.
Per esempio, dal 2014 ad oggi si è passati da circa 18 milioni di persone ad oltre 26 milioni di persone che si definiscono sviluppatori\urlfootnote{https://www.daxx.com/blog/development-trends/number-software-developers-world#:~:text=According%20to%20IDC%20calculations%2C%20in,only%2018%2C5%20million%20programmers}.
Inoltre, secondo \emph{Stack Overflow}, la stragrande maggioranza dei programmatori (soprattutto amatoriali) ha un'età inferiore ai 35 anni\urlfootnote{https://insights.stackoverflow.com/survey/2020#developer-profile-age-professional-developers5}.
Gli appartenenti a questa fascia di età sono anche i più interessati ai videogiochi\urlfootnote{https://www.earnest.com/blog/the-demographics-of-video-gaming/#:~:text=About%2016%25%20of%20applicants%20under,mobile%20gaming%20or%20console%20games}. 
Uno studio condotto nel Regno Unito mostra anche come sia in netta crescita il numero di laureati in corsi che hanno a che vedere con il \emph{game development}\urlfootnote{https://tiga.org/news/number-of-graduates-in-computer-games-continues-to-rise}. Inoltre, la velocità sempre crescente delle connessioni internet ha
permesso ai servizi di streaming di avere sempre più successo.
Per soddisfare le esigenze evidenti dai dati appena riportati, attualmente esistono molte piattaforme in cui un utente
può scegliere un gioco e giocarci liberamente, ed esistono anche
piattaforme che permettono di caricare i propri giochi. Queste
piattaforme richiedono però di avere dei requisiti particolari.

In questo contesto nasce l'idea di questa tesi.
Il progetto ha come obiettivo la creazione di una piattaforma per la
condivisione di videogiochi. Gli utenti di questa piattaforma possono
condividere con gli altri utenti i giochi da loro sviluppati. Tutti i
giochi caricati sulla piattaforma sono giocabili da ogni utente.
Infine, come utente, è possibile condividere le proprie esperienze di
gioco votando, commentando, invitando amici e altro.\\

La tesi si concentrerà su vari aspetti della realizzazione del progetto che sono stati opportunamente divisi in capitoli.\\

Nel capitolo 2 verranno trattate le funzionalità specifiche offerte dalla piattaforma, come è stato organizzato il lavoro e quali servizi sono stati utilizzati.\cite{git}

Il capitolo 3 si concentrerà sulla fase di analisi della realtà d'interesse e sulle scelte effettuate per risolvere le relative problematiche.\cite{DB}

Il capitolo 4 mostrerà come le scelte effettuate nella fase di analisi hanno avuto un riscontro rispetto all'architettura di sviluppo di riferimento.\cite{ProgSoft}\cite{RoR}\cite{docror}\cite{saas}

Nel capitolo 5 si potrà finalmente comprendere come avviene l'utilizzo vero e proprio dell'applicazione.

Nel capitolo 6 verrà trattato in breve l'approccio al testing all'interno del progetto.\cite{test}

Infine nel capitolo 7 si darà uno sguardo alle possibilità future dell'autenticazione attraverso le cosiddette \emph{behavioral biometrics}.\cite{tdna}
